% Options for packages loaded elsewhere
\PassOptionsToPackage{unicode}{hyperref}
\PassOptionsToPackage{hyphens}{url}
%
\documentclass[
]{article}
\usepackage{lmodern}
\usepackage{amssymb,amsmath}
\usepackage{ifxetex,ifluatex}
\ifnum 0\ifxetex 1\fi\ifluatex 1\fi=0 % if pdftex
  \usepackage[T1]{fontenc}
  \usepackage[utf8]{inputenc}
  \usepackage{textcomp} % provide euro and other symbols
\else % if luatex or xetex
  \usepackage{unicode-math}
  \defaultfontfeatures{Scale=MatchLowercase}
  \defaultfontfeatures[\rmfamily]{Ligatures=TeX,Scale=1}
\fi
% Use upquote if available, for straight quotes in verbatim environments
\IfFileExists{upquote.sty}{\usepackage{upquote}}{}
\IfFileExists{microtype.sty}{% use microtype if available
  \usepackage[]{microtype}
  \UseMicrotypeSet[protrusion]{basicmath} % disable protrusion for tt fonts
}{}
\makeatletter
\@ifundefined{KOMAClassName}{% if non-KOMA class
  \IfFileExists{parskip.sty}{%
    \usepackage{parskip}
  }{% else
    \setlength{\parindent}{0pt}
    \setlength{\parskip}{6pt plus 2pt minus 1pt}}
}{% if KOMA class
  \KOMAoptions{parskip=half}}
\makeatother
\usepackage{xcolor}
\IfFileExists{xurl.sty}{\usepackage{xurl}}{} % add URL line breaks if available
\IfFileExists{bookmark.sty}{\usepackage{bookmark}}{\usepackage{hyperref}}
\hypersetup{
  pdftitle={Introduction},
  hidelinks,
  pdfcreator={LaTeX via pandoc}}
\urlstyle{same} % disable monospaced font for URLs
\usepackage[margin=1in]{geometry}
\usepackage{color}
\usepackage{fancyvrb}
\newcommand{\VerbBar}{|}
\newcommand{\VERB}{\Verb[commandchars=\\\{\}]}
\DefineVerbatimEnvironment{Highlighting}{Verbatim}{commandchars=\\\{\}}
% Add ',fontsize=\small' for more characters per line
\usepackage{framed}
\definecolor{shadecolor}{RGB}{248,248,248}
\newenvironment{Shaded}{\begin{snugshade}}{\end{snugshade}}
\newcommand{\AlertTok}[1]{\textcolor[rgb]{0.94,0.16,0.16}{#1}}
\newcommand{\AnnotationTok}[1]{\textcolor[rgb]{0.56,0.35,0.01}{\textbf{\textit{#1}}}}
\newcommand{\AttributeTok}[1]{\textcolor[rgb]{0.77,0.63,0.00}{#1}}
\newcommand{\BaseNTok}[1]{\textcolor[rgb]{0.00,0.00,0.81}{#1}}
\newcommand{\BuiltInTok}[1]{#1}
\newcommand{\CharTok}[1]{\textcolor[rgb]{0.31,0.60,0.02}{#1}}
\newcommand{\CommentTok}[1]{\textcolor[rgb]{0.56,0.35,0.01}{\textit{#1}}}
\newcommand{\CommentVarTok}[1]{\textcolor[rgb]{0.56,0.35,0.01}{\textbf{\textit{#1}}}}
\newcommand{\ConstantTok}[1]{\textcolor[rgb]{0.00,0.00,0.00}{#1}}
\newcommand{\ControlFlowTok}[1]{\textcolor[rgb]{0.13,0.29,0.53}{\textbf{#1}}}
\newcommand{\DataTypeTok}[1]{\textcolor[rgb]{0.13,0.29,0.53}{#1}}
\newcommand{\DecValTok}[1]{\textcolor[rgb]{0.00,0.00,0.81}{#1}}
\newcommand{\DocumentationTok}[1]{\textcolor[rgb]{0.56,0.35,0.01}{\textbf{\textit{#1}}}}
\newcommand{\ErrorTok}[1]{\textcolor[rgb]{0.64,0.00,0.00}{\textbf{#1}}}
\newcommand{\ExtensionTok}[1]{#1}
\newcommand{\FloatTok}[1]{\textcolor[rgb]{0.00,0.00,0.81}{#1}}
\newcommand{\FunctionTok}[1]{\textcolor[rgb]{0.00,0.00,0.00}{#1}}
\newcommand{\ImportTok}[1]{#1}
\newcommand{\InformationTok}[1]{\textcolor[rgb]{0.56,0.35,0.01}{\textbf{\textit{#1}}}}
\newcommand{\KeywordTok}[1]{\textcolor[rgb]{0.13,0.29,0.53}{\textbf{#1}}}
\newcommand{\NormalTok}[1]{#1}
\newcommand{\OperatorTok}[1]{\textcolor[rgb]{0.81,0.36,0.00}{\textbf{#1}}}
\newcommand{\OtherTok}[1]{\textcolor[rgb]{0.56,0.35,0.01}{#1}}
\newcommand{\PreprocessorTok}[1]{\textcolor[rgb]{0.56,0.35,0.01}{\textit{#1}}}
\newcommand{\RegionMarkerTok}[1]{#1}
\newcommand{\SpecialCharTok}[1]{\textcolor[rgb]{0.00,0.00,0.00}{#1}}
\newcommand{\SpecialStringTok}[1]{\textcolor[rgb]{0.31,0.60,0.02}{#1}}
\newcommand{\StringTok}[1]{\textcolor[rgb]{0.31,0.60,0.02}{#1}}
\newcommand{\VariableTok}[1]{\textcolor[rgb]{0.00,0.00,0.00}{#1}}
\newcommand{\VerbatimStringTok}[1]{\textcolor[rgb]{0.31,0.60,0.02}{#1}}
\newcommand{\WarningTok}[1]{\textcolor[rgb]{0.56,0.35,0.01}{\textbf{\textit{#1}}}}
\usepackage{graphicx,grffile}
\makeatletter
\def\maxwidth{\ifdim\Gin@nat@width>\linewidth\linewidth\else\Gin@nat@width\fi}
\def\maxheight{\ifdim\Gin@nat@height>\textheight\textheight\else\Gin@nat@height\fi}
\makeatother
% Scale images if necessary, so that they will not overflow the page
% margins by default, and it is still possible to overwrite the defaults
% using explicit options in \includegraphics[width, height, ...]{}
\setkeys{Gin}{width=\maxwidth,height=\maxheight,keepaspectratio}
% Set default figure placement to htbp
\makeatletter
\def\fps@figure{htbp}
\makeatother
\setlength{\emergencystretch}{3em} % prevent overfull lines
\providecommand{\tightlist}{%
  \setlength{\itemsep}{0pt}\setlength{\parskip}{0pt}}
\setcounter{secnumdepth}{-\maxdimen} % remove section numbering

\title{Introduction}
\author{}
\date{\vspace{-2.5em}}

\begin{document}
\maketitle

\hypertarget{overview}{%
\subsection{Overview}\label{overview}}

This package allows you to build and run a Metapopulation Assessment
System (MAS) stock assessment model directly from R. It also includes
functions to inter-translate between different assessment platforms. In
this vignette, we detail how to set up a MAS model and run it from R
using dummy data included in the package. It is straightforward to
replace this with your own data.

The first step is to ensure you have \texttt{Rcpp}, the R package that
allows calling of C++ code directly from R, and \texttt{jsonlite}, the R
package that facilitates reading input and output files from JSON
format. Right now, the \texttt{RMAS} package is only available on Github
so you need to install it with the \texttt{remotes} package (this
package is included in \texttt{devtools} if you have already installed
that.)

\begin{Shaded}
\begin{Highlighting}[]
\CommentTok{##install.packages("Rcpp")}
\CommentTok{##install.packages("jsonlite")}
\CommentTok{##install.packages("remotes")}

\NormalTok{remotes}\OperatorTok{::}\KeywordTok{install_github}\NormalTok{(}\StringTok{"nmfs-fish-tools/RMAS"}\NormalTok{)}
\end{Highlighting}
\end{Shaded}

Next, you will need to load the .dll included in the RMAS package.

\begin{Shaded}
\begin{Highlighting}[]
\KeywordTok{require}\NormalTok{(Rcpp)}
\end{Highlighting}
\end{Shaded}

\begin{verbatim}
## Loading required package: Rcpp
\end{verbatim}

\begin{Shaded}
\begin{Highlighting}[]
\KeywordTok{require}\NormalTok{(RMAS)}
\end{Highlighting}
\end{Shaded}

\begin{verbatim}
## Loading required package: RMAS
\end{verbatim}

\begin{Shaded}
\begin{Highlighting}[]
\NormalTok{r4mas <-}
\StringTok{  }\KeywordTok{Module}\NormalTok{(}\StringTok{"rmas"}\NormalTok{, }\KeywordTok{dyn.load}\NormalTok{(}\KeywordTok{paste}\NormalTok{(}\StringTok{"src/RMAS"}\NormalTok{, .Platform}\OperatorTok{$}\NormalTok{dynlib.ext, }\DataTypeTok{sep =} \StringTok{""}\NormalTok{)))}
\end{Highlighting}
\end{Shaded}

\hypertarget{setting-model-specifications}{%
\subsection{Setting model
specifications}\label{setting-model-specifications}}

Next, we define model scalars, such as the number of years of the model,
the number of seasons, and the vector of ages. For this example, we will
use the test data created by \texttt{rmas::write\_test\_data()}, but
this can be replaced by user input data later on in the script.

RMAS uses classes to parallel the object-oriented structure of MAS. In
R, model classes are initiralized using the \texttt{new()} command. For
each model component, the associated class is initialized and then its
attributes are populated. To view available attributes for each class,
use the \texttt{show()} function as shown below.

\begin{Shaded}
\begin{Highlighting}[]
\CommentTok{#View attributes and methods associated with each class}
\KeywordTok{show}\NormalTok{(r4mas}\OperatorTok{$}\NormalTok{Area)}
\end{Highlighting}
\end{Shaded}

\begin{verbatim}
## C++ class 'Area' <0000000014ba7a80>
## Constructors:
##     Area()
## 
## Fields: 
##     int id
##     std::string name
## 
## Methods: no methods exposed by this class
\end{verbatim}

\begin{Shaded}
\begin{Highlighting}[]
\CommentTok{#define area}
\NormalTok{area1<-}\KeywordTok{new}\NormalTok{(r4mas}\OperatorTok{$}\NormalTok{Area)}
\NormalTok{area1}\OperatorTok{$}\NormalTok{name<-}\StringTok{"area1"}
\end{Highlighting}
\end{Shaded}

\hypertarget{setting-parameter-values}{%
\subsection{Setting parameter values}\label{setting-parameter-values}}

Next, we populate the parameter constructors for each different set of
parameters: recruitment, growth, maturity, mortality, and movement. For
functions with several parameters (e.g.~the Beverton-Holt recruitment
function), the \texttt{create\_par\_section()} function exists. This
takes a number of arguments: the type of section, the associated object
created for that function, then a series of vectors of length equivalent
to the number of parameters. These vectors denote the name, lower bound,
upper bound, units, phase, and value for each parameter. If any
parameter doesn't have one of these attributes, you can use an NA in the
vector. Or, if one of the attributes is not defined for all attributes
(e.g.~units in this case), you can pass a single NA for that attribute
or leave the argument blank.

\begin{Shaded}
\begin{Highlighting}[]
\CommentTok{#Recruitment}
\NormalTok{recruitment<-}\KeywordTok{new}\NormalTok{(r4mas}\OperatorTok{$}\NormalTok{BevertonHoltRecruitment)}

\NormalTok{devs_list <-}\StringTok{ }\KeywordTok{list}\NormalTok{(}\OtherTok{TRUE}\NormalTok{, }\OtherTok{TRUE}\NormalTok{, }\KeywordTok{rep}\NormalTok{(}\FloatTok{0.0}\NormalTok{,}\DecValTok{22}\NormalTok{))}
\NormalTok{recruitment <-}\StringTok{ }\KeywordTok{create_par_section}\NormalTok{(}\DataTypeTok{section_type =} \StringTok{"recruitment"}\NormalTok{, }\DataTypeTok{section_type_object =}\NormalTok{ recruitment, }\DataTypeTok{par_names =} \KeywordTok{c}\NormalTok{(}\StringTok{"R0"}\NormalTok{,}\StringTok{"h"}\NormalTok{,}\StringTok{"sigma_r"}\NormalTok{,}\StringTok{"recdevs"}\NormalTok{),}
                   \DataTypeTok{par_lo =} \KeywordTok{c}\NormalTok{(}\OtherTok{NA}\NormalTok{, }\FloatTok{0.2001}\NormalTok{,}\OtherTok{NA}\NormalTok{,}\OperatorTok{-}\DecValTok{15}\NormalTok{), }\DataTypeTok{par_hi =} \KeywordTok{c}\NormalTok{(}\OtherTok{NA}\NormalTok{,}\FloatTok{1.0}\NormalTok{,}\OtherTok{NA}\NormalTok{,}\DecValTok{15}\NormalTok{), }\DataTypeTok{par_units =} \OtherTok{NA}\NormalTok{, }\DataTypeTok{par_phase =} \KeywordTok{c}\NormalTok{(}\DecValTok{1}\NormalTok{,}\OperatorTok{-}\DecValTok{2}\NormalTok{,}\OperatorTok{-}\DecValTok{1}\NormalTok{,}\DecValTok{1}\NormalTok{), }\DataTypeTok{par_value =} \KeywordTok{c}\NormalTok{(}\DecValTok{1000}\NormalTok{,}\FloatTok{0.75}\NormalTok{,}\FloatTok{0.55}\NormalTok{, }\OtherTok{NA}\NormalTok{), }
                   \DataTypeTok{rec_devs=}\NormalTok{devs_list)}


\CommentTok{#Initial Deviations}
\NormalTok{initial_deviations<-}\KeywordTok{new}\NormalTok{(r4mas}\OperatorTok{$}\NormalTok{InitialDeviations)}
\NormalTok{initial_deviations}\OperatorTok{$}\NormalTok{values<-}\KeywordTok{rep}\NormalTok{(}\FloatTok{0.0}\NormalTok{,nages)}
\NormalTok{initial_deviations}\OperatorTok{$}\NormalTok{estimate<-}\OtherTok{FALSE}
\NormalTok{initial_deviations}\OperatorTok{$}\NormalTok{phase<-}\DecValTok{1}


\CommentTok{#Growth}
\NormalTok{growth<-}\KeywordTok{new}\NormalTok{(r4mas}\OperatorTok{$}\NormalTok{VonBertalanffyModified)}
\NormalTok{empirical_weight <-}\StringTok{ }\NormalTok{input_data}\OperatorTok{$}\NormalTok{ewaa}
\NormalTok{growth}\OperatorTok{$}\KeywordTok{SetUndifferentiatedCatchWeight}\NormalTok{(empirical_weight)}
\NormalTok{growth}\OperatorTok{$}\KeywordTok{SetUndifferentiatedSurveyWeight}\NormalTok{(empirical_weight)}
\NormalTok{growth}\OperatorTok{$}\KeywordTok{SetUndifferentiatedWeightAtSeasonStart}\NormalTok{(empirical_weight)}
\NormalTok{growth}\OperatorTok{$}\KeywordTok{SetUndifferentiatedWeightAtSpawning}\NormalTok{(empirical_weight)}
\NormalTok{growth <-}\StringTok{ }\KeywordTok{create_par_section}\NormalTok{(}\DataTypeTok{section_type=}\StringTok{"growth"}\NormalTok{, }\DataTypeTok{section_type_object =}\NormalTok{ growth, }
                             \DataTypeTok{par_names =} \KeywordTok{c}\NormalTok{(}\StringTok{"a_min"}\NormalTok{,}\StringTok{"a_max"}\NormalTok{,}\StringTok{"c"}\NormalTok{,}\StringTok{"lmin"}\NormalTok{,}\StringTok{"lmax"}\NormalTok{,}\StringTok{"alpha_f"}\NormalTok{,}\StringTok{"alpha_m"}\NormalTok{,}\StringTok{"beta_f"}\NormalTok{,}\StringTok{"beta_m"}\NormalTok{),}
                             \DataTypeTok{par_value=}\KeywordTok{c}\NormalTok{(}\FloatTok{0.01}\NormalTok{,}\FloatTok{10.0}\NormalTok{,}\FloatTok{0.3}\NormalTok{,}\DecValTok{5}\NormalTok{,}\DecValTok{50}\NormalTok{,}\FloatTok{2.5E-5}\NormalTok{,}\FloatTok{2.5E-5}\NormalTok{,}\FloatTok{2.9624}\NormalTok{,}\FloatTok{2.9624}\NormalTok{))}

\CommentTok{#Maturity}
\NormalTok{maturity<-}\KeywordTok{new}\NormalTok{(r4mas}\OperatorTok{$}\NormalTok{Maturity)}
\NormalTok{maturity}\OperatorTok{$}\NormalTok{values<-}\KeywordTok{c}\NormalTok{(}\FloatTok{0.011488685}\NormalTok{,}
                   \FloatTok{0.16041065}\NormalTok{,}
                   \FloatTok{0.45232527}\NormalTok{,}
                   \FloatTok{0.497389935}\NormalTok{,}
                   \FloatTok{0.499869405}\NormalTok{,}
                   \FloatTok{0.499993495}\NormalTok{,}
                   \FloatTok{0.499999675}\NormalTok{,}
                   \FloatTok{0.499999985}\NormalTok{,}
                   \FloatTok{0.5}\NormalTok{,}
                   \FloatTok{0.5}\NormalTok{,}
                   \FloatTok{0.5}\NormalTok{,}
                   \FloatTok{0.5}\NormalTok{)}

\CommentTok{#Natural Mortality}
\NormalTok{natural_mortality<-}\KeywordTok{new}\NormalTok{(r4mas}\OperatorTok{$}\NormalTok{NaturalMortality)}
\NormalTok{natural_mortality}\OperatorTok{$}\KeywordTok{SetValues}\NormalTok{(}\KeywordTok{rep}\NormalTok{(}\FloatTok{0.2}\NormalTok{,nages))}
\CommentTok{#need to add reset button to static variable so you can run again}

\CommentTok{#define Movement (only 1 area in this model)}
\NormalTok{movement<-}\KeywordTok{new}\NormalTok{(r4mas}\OperatorTok{$}\NormalTok{Movement)}
\NormalTok{movement}\OperatorTok{$}\NormalTok{connectivity_females<-}\KeywordTok{c}\NormalTok{(}\FloatTok{1.0}\NormalTok{)}
\NormalTok{movement}\OperatorTok{$}\NormalTok{connectivity_males<-}\KeywordTok{c}\NormalTok{(}\FloatTok{1.0}\NormalTok{)}
\NormalTok{movement}\OperatorTok{$}\NormalTok{connectivity_recruits<-}\KeywordTok{c}\NormalTok{(}\FloatTok{1.0}\NormalTok{)}
\end{Highlighting}
\end{Shaded}

Next we define the functions related to fishing, i.e.~fishing mortality
and selectivity.

\begin{Shaded}
\begin{Highlighting}[]
\CommentTok{#Fishing Mortality}
\NormalTok{fishing_mortality<-}\KeywordTok{new}\NormalTok{(r4mas}\OperatorTok{$}\NormalTok{FishingMortality)}
\NormalTok{fishing_mortality}\OperatorTok{$}\NormalTok{estimate<-}\OtherTok{TRUE}
\NormalTok{fishing_mortality}\OperatorTok{$}\NormalTok{phase<-}\DecValTok{1}
\NormalTok{fishing_mortality}\OperatorTok{$}\NormalTok{min<-}\FloatTok{1e-8}
\NormalTok{fishing_mortality}\OperatorTok{$}\NormalTok{max<-}\DecValTok{30}
\NormalTok{fishing_mortality}\OperatorTok{$}\KeywordTok{SetValues}\NormalTok{(}\KeywordTok{rep}\NormalTok{(}\FloatTok{0.3}\NormalTok{,}\DecValTok{30}\NormalTok{))}

\CommentTok{#Selectivity Model}
\NormalTok{fleet_selectivity<-}\KeywordTok{new}\NormalTok{(r4mas}\OperatorTok{$}\NormalTok{LogisticSelectivity)}
\NormalTok{fleet_selectivity <-}\StringTok{ }\KeywordTok{create_par_section}\NormalTok{(}\StringTok{"selectivity"}\NormalTok{,fleet_selectivity, }\DataTypeTok{par_names =} \KeywordTok{c}\NormalTok{(}\StringTok{"a50"}\NormalTok{,}\StringTok{"slope"}\NormalTok{),}
                                        \DataTypeTok{par_lo =} \KeywordTok{c}\NormalTok{(}\FloatTok{0.0}\NormalTok{,}\FloatTok{0.0001}\NormalTok{), }\DataTypeTok{par_hi =} \KeywordTok{c}\NormalTok{(}\FloatTok{10.0}\NormalTok{,}\FloatTok{5.0}\NormalTok{), }\DataTypeTok{par_phase =} \KeywordTok{c}\NormalTok{(}\DecValTok{2}\NormalTok{,}\FloatTok{0.5}\NormalTok{),}
                                        \DataTypeTok{par_value =} \KeywordTok{c}\NormalTok{(}\FloatTok{1.5}\NormalTok{,}\FloatTok{1.5}\NormalTok{))}


\NormalTok{survey_selectivity<-}\KeywordTok{new}\NormalTok{(r4mas}\OperatorTok{$}\NormalTok{LogisticSelectivity)}
\NormalTok{survey_selectivity <-}\StringTok{ }\KeywordTok{create_par_section}\NormalTok{(}\StringTok{"selectivity"}\NormalTok{,survey_selectivity, }\DataTypeTok{par_names =} \KeywordTok{c}\NormalTok{(}\StringTok{"a50"}\NormalTok{,}\StringTok{"slope"}\NormalTok{),}
                                        \DataTypeTok{par_lo =} \KeywordTok{c}\NormalTok{(}\FloatTok{0.0}\NormalTok{,}\FloatTok{0.0001}\NormalTok{), }\DataTypeTok{par_hi =} \KeywordTok{c}\NormalTok{(}\FloatTok{10.0}\NormalTok{,}\FloatTok{5.0}\NormalTok{), }\DataTypeTok{par_phase =} \KeywordTok{c}\NormalTok{(}\DecValTok{2}\NormalTok{,}\DecValTok{2}\NormalTok{),}
                                        \DataTypeTok{par_value =} \KeywordTok{c}\NormalTok{(}\FloatTok{1.0}\NormalTok{,}\FloatTok{0.3}\NormalTok{))}


\NormalTok{survey2_selectivity<-}\KeywordTok{new}\NormalTok{(r4mas}\OperatorTok{$}\NormalTok{AgeBasedSelectivity)}
\NormalTok{survey2_selectivity}\OperatorTok{$}\NormalTok{estimated<-}\OtherTok{FALSE}
\NormalTok{survey2_selectivity}\OperatorTok{$}\NormalTok{values<-}\KeywordTok{c}\NormalTok{(}\FloatTok{1.0}\NormalTok{,}\KeywordTok{rep}\NormalTok{(}\FloatTok{0.0}\NormalTok{,nages}\DecValTok{-1}\NormalTok{))}
\end{Highlighting}
\end{Shaded}

\hypertarget{creating-the-population}{%
\subsection{Creating the population}\label{creating-the-population}}

\begin{Shaded}
\begin{Highlighting}[]
\NormalTok{population<-}\KeywordTok{new}\NormalTok{(r4mas}\OperatorTok{$}\NormalTok{Population)}

\ControlFlowTok{for}\NormalTok{ (y }\ControlFlowTok{in} \DecValTok{0}\OperatorTok{:}\NormalTok{(nyears))}
\NormalTok{\{}
\NormalTok{  population}\OperatorTok{$}\KeywordTok{AddMovement}\NormalTok{(movement}\OperatorTok{$}\NormalTok{id, y)}
\NormalTok{\}}

\NormalTok{population}\OperatorTok{$}\KeywordTok{AddNaturalMortality}\NormalTok{(natural_mortality}\OperatorTok{$}\NormalTok{id,area1}\OperatorTok{$}\NormalTok{id,}\StringTok{"undifferentiated"}\NormalTok{)}
\NormalTok{population}\OperatorTok{$}\KeywordTok{AddMaturity}\NormalTok{(maturity}\OperatorTok{$}\NormalTok{id,area1}\OperatorTok{$}\NormalTok{id, }\StringTok{"undifferentiated"}\NormalTok{)}
\NormalTok{population}\OperatorTok{$}\KeywordTok{AddRecruitment}\NormalTok{(recruitment}\OperatorTok{$}\NormalTok{id,area1}\OperatorTok{$}\NormalTok{id)}
\NormalTok{population}\OperatorTok{$}\KeywordTok{SetInitialDeviations}\NormalTok{(initial_deviations}\OperatorTok{$}\NormalTok{id, area1}\OperatorTok{$}\NormalTok{id, }\StringTok{"undifferentiated"}\NormalTok{)}
\NormalTok{population}\OperatorTok{$}\KeywordTok{SetGrowth}\NormalTok{(growth}\OperatorTok{$}\NormalTok{id)}
\NormalTok{population}\OperatorTok{$}\NormalTok{sex_ratio<-}\FloatTok{0.5}
\end{Highlighting}
\end{Shaded}

\hypertarget{adding-data-to-the-model}{%
\subsection{Adding data to the model}\label{adding-data-to-the-model}}

\begin{Shaded}
\begin{Highlighting}[]
\CommentTok{#Index data}
\NormalTok{catch_index<-}\KeywordTok{new}\NormalTok{(r4mas}\OperatorTok{$}\NormalTok{IndexData)}
\NormalTok{catch_index}\OperatorTok{$}\NormalTok{values <-}\StringTok{ }\NormalTok{input_data}\OperatorTok{$}\NormalTok{catch_index}\OperatorTok{$}\NormalTok{values}
\NormalTok{catch_index}\OperatorTok{$}\NormalTok{error <-}\StringTok{ }\NormalTok{input_data}\OperatorTok{$}\NormalTok{catch_index}\OperatorTok{$}\NormalTok{error}

\NormalTok{survey_index<-}\KeywordTok{new}\NormalTok{(r4mas}\OperatorTok{$}\NormalTok{IndexData)}
\NormalTok{survey_index}\OperatorTok{$}\NormalTok{values <-}\StringTok{ }\NormalTok{input_data}\OperatorTok{$}\NormalTok{survey_index}\OperatorTok{$}\NormalTok{values}
\NormalTok{survey_index}\OperatorTok{$}\NormalTok{error <-}\StringTok{ }\NormalTok{input_data}\OperatorTok{$}\NormalTok{survey_index}\OperatorTok{$}\NormalTok{error}


\CommentTok{#Age Comp Data}
\NormalTok{catch_comp<-}\KeywordTok{new}\NormalTok{(r4mas}\OperatorTok{$}\NormalTok{AgeCompData)}
\NormalTok{catch_comp}\OperatorTok{$}\NormalTok{values <-}\StringTok{ }\NormalTok{input_data}\OperatorTok{$}\NormalTok{catch_comp}\OperatorTok{$}\NormalTok{values}
\NormalTok{catch_comp}\OperatorTok{$}\NormalTok{sample_size <-input_data}\OperatorTok{$}\NormalTok{catch_comp}\OperatorTok{$}\NormalTok{sample_size}



\NormalTok{survey_comp<-}\KeywordTok{new}\NormalTok{(r4mas}\OperatorTok{$}\NormalTok{AgeCompData)}
\NormalTok{survey_comp}\OperatorTok{$}\NormalTok{values <-}\StringTok{ }\NormalTok{input_data}\OperatorTok{$}\NormalTok{survey_comp}\OperatorTok{$}\NormalTok{values}
\NormalTok{survey_comp}\OperatorTok{$}\NormalTok{sample_size <-input_data}\OperatorTok{$}\NormalTok{survey_comp}\OperatorTok{$}\NormalTok{sample_size}
\end{Highlighting}
\end{Shaded}

\hypertarget{defining-fleets}{%
\subsection{Defining fleets}\label{defining-fleets}}

\begin{Shaded}
\begin{Highlighting}[]
\CommentTok{#NLL models}
\NormalTok{fleet_index_comp_nll<-survey_index_comp_nll <-}\StringTok{ }\NormalTok{survey2_index_comp_nll <-}\StringTok{ }\KeywordTok{new}\NormalTok{(r4mas}\OperatorTok{$}\NormalTok{Lognormal)}
\NormalTok{fleet_age_comp_nll<-survey_age_comp_nll <-}\StringTok{ }\KeywordTok{new}\NormalTok{(r4mas}\OperatorTok{$}\NormalTok{MultinomialRobust)}


\CommentTok{#Fleet}
\NormalTok{fleet<-}\StringTok{ }\KeywordTok{make_fleet}\NormalTok{(r4mas, catch_comp, catch_index, fleet_selectivity, area1, fishing_mortality)}

\CommentTok{#Survey}
\NormalTok{survey<-}\KeywordTok{new}\NormalTok{(r4mas}\OperatorTok{$}\NormalTok{Survey)}
\NormalTok{survey}\OperatorTok{$}\KeywordTok{AddAgeCompData}\NormalTok{(survey_comp}\OperatorTok{$}\NormalTok{id,}\StringTok{"undifferentiated"}\NormalTok{)}
\NormalTok{survey}\OperatorTok{$}\KeywordTok{AddIndexData}\NormalTok{(survey_index}\OperatorTok{$}\NormalTok{id,}\StringTok{"undifferentiated"}\NormalTok{)}
\NormalTok{survey}\OperatorTok{$}\KeywordTok{SetIndexNllComponent}\NormalTok{(survey_index_comp_nll}\OperatorTok{$}\NormalTok{id)}
\NormalTok{survey}\OperatorTok{$}\KeywordTok{SetAgeCompNllComponent}\NormalTok{(survey_age_comp_nll}\OperatorTok{$}\NormalTok{id)}
\NormalTok{survey}\OperatorTok{$}\KeywordTok{AddSelectivity}\NormalTok{(survey_selectivity}\OperatorTok{$}\NormalTok{id, }\DecValTok{1}\NormalTok{, area1}\OperatorTok{$}\NormalTok{id)}
\NormalTok{survey}\OperatorTok{$}\NormalTok{q}\OperatorTok{$}\NormalTok{value<-}\FloatTok{0.0001}
\NormalTok{survey}\OperatorTok{$}\NormalTok{q}\OperatorTok{$}\NormalTok{min<-}\DecValTok{0}
\NormalTok{survey}\OperatorTok{$}\NormalTok{q}\OperatorTok{$}\NormalTok{max<-}\DecValTok{10}
\NormalTok{survey}\OperatorTok{$}\NormalTok{q}\OperatorTok{$}\NormalTok{estimated<-}\OtherTok{TRUE}
\NormalTok{survey}\OperatorTok{$}\NormalTok{q}\OperatorTok{$}\NormalTok{phase<-}\DecValTok{1}
\end{Highlighting}
\end{Shaded}

\hypertarget{build-and-run-the-model-and-write-output}{%
\subsection{Build and run the model and write
output}\label{build-and-run-the-model-and-write-output}}

\begin{Shaded}
\begin{Highlighting}[]
\CommentTok{#build the MAS model}
\NormalTok{mas_model<-}\KeywordTok{new}\NormalTok{(r4mas}\OperatorTok{$}\NormalTok{MASModel)}
\NormalTok{mas_model}\OperatorTok{$}\NormalTok{nyears<-nyears}
\NormalTok{mas_model}\OperatorTok{$}\NormalTok{nseasons<-nseasons}
\NormalTok{mas_model}\OperatorTok{$}\NormalTok{nages<-nages}
\NormalTok{mas_model}\OperatorTok{$}\NormalTok{extended_plus_group<-}\DecValTok{15}
\NormalTok{mas_model}\OperatorTok{$}\NormalTok{ages<-ages}
\NormalTok{mas_model}\OperatorTok{$}\KeywordTok{AddFleet}\NormalTok{(fleet}\OperatorTok{$}\NormalTok{id)}
\NormalTok{mas_model}\OperatorTok{$}\NormalTok{catch_season_offset<-}\FloatTok{0.5}
\NormalTok{mas_model}\OperatorTok{$}\NormalTok{spawning_season_offset<-}\FloatTok{0.5}

\NormalTok{mas_model}\OperatorTok{$}\KeywordTok{AddSurvey}\NormalTok{(survey}\OperatorTok{$}\NormalTok{id)}
\NormalTok{mas_model}\OperatorTok{$}\KeywordTok{AddPopulation}\NormalTok{(population}\OperatorTok{$}\NormalTok{id)}


\CommentTok{#########################################################}
\CommentTok{# Run the model}
\CommentTok{############################################################}

\NormalTok{mas_model}\OperatorTok{$}\KeywordTok{Run}\NormalTok{()}
\NormalTok{mas_model}
\end{Highlighting}
\end{Shaded}

\begin{verbatim}
## C++ object <0000000016c1bc20> of class 'MASModel' <0000000016c15900>
\end{verbatim}

\begin{Shaded}
\begin{Highlighting}[]
\KeywordTok{write}\NormalTok{(mas_model}\OperatorTok{$}\KeywordTok{GetOutput}\NormalTok{(), }\DataTypeTok{file=}\StringTok{"mas_s2_output.json"}\NormalTok{)}
\end{Highlighting}
\end{Shaded}

\end{document}
